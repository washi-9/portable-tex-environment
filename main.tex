% 1. クラス宣言(LuaLaTeX用に変更)
\documentclass[autodetect-engine, ja=standard, a4paper]{bxjsarticle}

% 2. 以前の便利なパッケージ群(そのまま活用)
\geometry{top=20mm,bottom=20mm,left=20mm,right=20mm}
\usepackage{subcaption}
\usepackage{comment}
\usepackage{graphicx}
\usepackage{url}
\usepackage{xcolor}
\usepackage{amsmath}
\usepackage{float}

\usepackage{listings}
\lstset{
  stringstyle={\ttfamily},
  % inputencoding=utf8, % LuaLaTeXでは標準がUTF-8なので不要
  commentstyle={\ttfamily\itshape\color{blue}},
  basicstyle=\ttfamily,
  columns=[l]{fullflexible},
  frame={tb,leftline,rightline},
  breaklines=true,
  backgroundcolor=\color{gray!14},
  numbers=left,
  numberstyle={\scriptsize},
  stepnumber=1,
  numbersep=5pt,
  tabsize=2,
  lineskip=-0.6ex,
  keepspaces=true, % スペースを保持
}

% 3. ドキュメント開始
\title{新しいLaTeX環境のテスト}
\author{washi-9}
\begin{document}
\maketitle

\section{はじめに}
このコンテナは環境構築不要で、LuaLaTeXを使った日本語LaTeX文書の作成をサポートします。

\section{コードのテスト}
LuaLaTeXなら、以下のように日本語コメントを含むコードもきれいに表示されます。

\begin{lstlisting}[caption=テストコード, language=C++]
#include <iostream>
using namespace std;

int main() {
    // 日本語のコメントも文字化けしません
    cout << "Hello, LuaLaTeX!" << endl;
    return 0;
}
\end{lstlisting}

  \clearpage
  \begin{thebibliography}{99}

  \bibitem{test} test.
  \end{thebibliography}

\end{document}